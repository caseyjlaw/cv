%%%%%%%%%%%%%%%%%%%%%%%%%%%%%%%%%%%%%%%%%%%%%%%%%%%%%%%%%%%%%%%%%%%%%%%%
%%%%%%%%%%%%%%%%%%%%%% Simple LaTeX CV Template %%%%%%%%%%%%%%%%%%%%%%%%
%%%%%%%%%%%%%%%%%%%%%%%%%%%%%%%%%%%%%%%%%%%%%%%%%%%%%%%%%%%%%%%%%%%%%%%%

%%%%%%%%%%%%%%%%%%%%%%%%%%%% Document Setup %%%%%%%%%%%%%%%%%%%%%%%%%%%%

% Don't like 10pt? Try 11pt or 12pt
\documentclass[11pt]{article}
\RequirePackage[T1]{fontenc}

% This is a helpful package that puts math inside length specifications
\usepackage{calc}

% This package helps LaTeX auto-hyphenate hyphenated words if you use
% special hyphens. For example, bio\-/mimicry will properly hyphenate
% ``mimicry'' if necessary.
\usepackage[shortcuts]{extdash}

% Layout: Puts the section titles on left side of page
\reversemarginpar

%% Use these lines for letter-sized paper
\usepackage[paper=letterpaper,
            %includefoot, % Uncomment to put page number above margin
            marginparwidth=1.2in,     % Length of section titles
            marginparsep=.05in,       % Space between titles and text
            margin=1in,               % 1 inch margins
            includemp]{geometry}

%% More layout: Get rid of indenting throughout entire document
\setlength{\parindent}{0in}

% Provides special list environments and macros to create new ones
\usepackage[shortlabels]{enumitem}

% Simpler bibsections for CV sections
% (thanks to natbib for inspiration)
%
% * For lists of references with hanging indents and no numbers:
%
%   \begin{bibsection}
%       \item ...
%   \end{bibsection}
%
% * For numbered lists of references (with hanging indents):
%
%   \begin{bibenum}
%       \item ...
%   \end{bibenum}
%
%   Note that bibenum numbers continuously throughout. To reset the
%   counter, use
%
%   \restartlist{bibenum}
%
%   at the place where you want the numbering to reset.

\makeatletter
\newlength{\bibhang}
\setlength{\bibhang}{1em}
\newlength{\bibsep}
 {\@listi \global\bibsep\itemsep \global\advance\bibsep by\parsep}
\newlist{bibsection}{itemize}{3}
\setlist[bibsection]{label=,leftmargin=\bibhang,%
        itemindent=-\bibhang,
        itemsep=\bibsep,parsep=\z@,partopsep=0pt,
        topsep=0pt}
\newlist{bibenum}{enumerate}{3}
\setlist[bibenum]{label=[\arabic*],resume,leftmargin={\bibhang+\widthof{[999]}},%
        itemindent=-\bibhang,
        itemsep=\bibsep,parsep=\z@,partopsep=0pt,
        topsep=0pt}
\let\oldendbibenum\endbibenum
\def\endbibenum{\oldendbibenum\vspace{-.6\baselineskip}}
\let\oldendbibsection\endbibsection
\def\endbibsection{\oldendbibsection\vspace{-.6\baselineskip}}
\makeatother

%% Reference the last page in the page number
%
% NOTE: comment the +LP line and uncomment the -LP line to have page
%       numbers without the ``of ##'' last page reference)
%
% NOTE: uncomment the \pagestyle{empty} line to get rid of all page
%       numbers (make sure includefoot is commented out above)
%
\usepackage{fancyhdr,lastpage}
\pagestyle{fancy}
%\pagestyle{empty}      % Uncomment this to get rid of page numbers
\fancyhf{}\renewcommand{\headrulewidth}{0pt}
\fancyfootoffset{\marginparsep+\marginparwidth}
\newlength{\footpageshift}
\setlength{\footpageshift}
          {0.5\textwidth+0.5\marginparsep+0.5\marginparwidth-2in}
\lfoot{\hspace{\footpageshift}%
       \parbox{4in}{\, \hfill %
                    \arabic{page} of \protect\pageref*{LastPage} % +LP
%                    \arabic{page}                               % -LP
                    \hfill \,}}

% Finally, give us PDF bookmarks
\usepackage{color,hyperref}
\definecolor{darkblue}{rgb}{0.0,0.0,0.3}
\hypersetup{colorlinks,breaklinks,
            linkcolor=darkblue,urlcolor=darkblue,
            anchorcolor=darkblue,citecolor=darkblue}

%%%%%%%%%%%%%%%%%%%%%%%% End Document Setup %%%%%%%%%%%%%%%%%%%%%%%%%%%%


%%%%%%%%%%%%%%%%%%%%%%%%%%% Helper Commands %%%%%%%%%%%%%%%%%%%%%%%%%%%%

%%% HEADING AT TOP OF CURRICULUM VITAE

% The title (name) with a horizontal rule under it
% (optional argument typesets an object right-justified across from name
%  as well)
%
% Usage: \makeheading{name}
%        OR
%        \makeheading[right_object]{name}
%
% Place at top of document. It should be the first thing.
% If ``right_object'' is provided in the square-braced optional
% argument, it will be right justified on the same line as ``name'' at
% the top of the CV. For example:
%
%       \makeheading[\emph{Curriculum vitae}]{Your Name}
%
% will put an emphasized ``Curriculum vitae'' at the top of the document
% as a title. Likewise, a picture could be included:
%
%   \makeheading[{\includegraphics[height=1.5in]{my_picture}}]{Your Name}
%
% the picture will be flush right across from the name. For this example
% to work, make sure the extra set of curly braces is included. Also
% makes ure that \usepackage{graphicx} is somewhere in the preamble.
\newcommand{\makeheading}[2][]%
        {\hspace*{-\marginparsep minus \marginparwidth}%
         \begin{minipage}[t]{\textwidth+\marginparwidth+\marginparsep}%
             {\large \bfseries #2 \hfill #1}\\[-0.15\baselineskip]%
                 \rule{\columnwidth}{1pt}%
         \end{minipage}}

%%% SECTION HEADINGS

% The section headings. Flush left in small caps down pseudo-margin.
%
% Usage: \section{section name}
\renewcommand{\section}[1]{\pagebreak[3]%
    \vspace{1.3\baselineskip}%
    \phantomsection\addcontentsline{toc}{section}{#1}%
    \noindent\llap{\scshape\smash{\parbox[t]{\marginparwidth}{\hyphenpenalty=10000\raggedright #1}}}%
    \vspace{-\baselineskip}\par}

%%% LISTS

% This macro alters a list by removing some of the space that follows the list
% (is used by lists below)
\newcommand*\fixendlist[1]{%
    \expandafter\let\csname preFixEndListend#1\expandafter\endcsname\csname end#1\endcsname
    \expandafter\def\csname end#1\endcsname{\csname preFixEndListend#1\endcsname\vspace{-0.6\baselineskip}}}

% These macros help ensure that items in outer-type lists do not get
% separated from the next line by a page break
% (they are used by lists below)
\let\originalItem\item
\newcommand*\fixouterlist[1]{%
    \expandafter\let\csname preFixOuterList#1\expandafter\endcsname\csname #1\endcsname
    \expandafter\def\csname #1\endcsname{\let\oldItem\item\def\item{\pagebreak[2]\oldItem}\csname preFixOuterList#1\endcsname}
    \expandafter\let\csname preFixOuterListend#1\expandafter\endcsname\csname end#1\endcsname
    \expandafter\def\csname end#1\endcsname{\let\item\oldItem\csname preFixOuterListend#1\endcsname}}
\newcommand*\fixinnerlist[1]{%
    \expandafter\let\csname preFixInnerList#1\expandafter\endcsname\csname #1\endcsname
    \expandafter\def\csname #1\endcsname{\let\oldItem\item\let\item\originalItem\csname preFixInnerList#1\endcsname}
    \expandafter\let\csname preFixInnerListend#1\expandafter\endcsname\csname end#1\endcsname
    \expandafter\def\csname end#1\endcsname{\csname preFixInnerListend#1\endcsname\let\item\oldItem}}

% An itemize-style list with lots of space between items
%
% Usage:
%   \begin{outerlist}
%       \item ...    % (or \item[] for no bullet)
%   \end{outerlist}
\newlist{outerlist}{itemize}{3}
    \setlist[outerlist]{label=\enskip\textbullet,leftmargin=*}
    \fixendlist{outerlist}
    \fixouterlist{outerlist}

% An environment IDENTICAL to outerlist that has better pre-list spacing
% when used as the first thing in a \section
%
% Usage:
%   \begin{lonelist}
%       \item ...    % (or \item[] for no bullet)
%   \end{lonelist}
\newlist{lonelist}{itemize}{3}
    \setlist[lonelist]{label=\enskip\textbullet,leftmargin=*,partopsep=0pt,topsep=0pt}
    \fixendlist{lonelist}
    \fixouterlist{lonelist}

% An itemize-style list with little space between items
%
% Usage:
%   \begin{innerlist}
%       \item ...    % (or \item[] for no bullet)
%   \end{innerlist}
\newlist{innerlist}{itemize}{3}
    \setlist[innerlist]{label=\enskip\textbullet,leftmargin=*,parsep=0pt,itemsep=0pt,topsep=0pt,partopsep=0pt}
    \fixinnerlist{innerlist}

% An environment IDENTICAL to innerlist that has better pre-list spacing
% when used as the first thing in a \section
%
% Usage:
%   \begin{loneinnerlist}
%       \item ...    % (or \item[] for no bullet)
%   \end{loneinnerlist}
\newlist{loneinnerlist}{itemize}{3}
    \setlist[loneinnerlist]{label=\enskip\textbullet,leftmargin=*,parsep=0pt,itemsep=0pt,topsep=0pt,partopsep=0pt}
    \fixendlist{loneinnerlist}
    \fixinnerlist{loneinnerlist}

%%% EXTRA SPACE

% To add some paragraph space between lines.
% This also tells LaTeX to preferably break a page on one of these gaps
% if there is a needed pagebreak nearby.
\newcommand{\blankline}{\quad\pagebreak[3]}
\newcommand{\halfblankline}{\quad\vspace{-0.5\baselineskip}\pagebreak[3]}

%%% FORMATTING MACROS

% Provides a linked \doi{#1} that links doi:#1 to http://dx.doi.org/#1
\usepackage{doi}
% To change the text before the DOI, adjust this command
%\renewcommand\doitext{doi:}

% Provides a linked \url{#1} that doesn't require escape characters
\usepackage{url}

% You can adjust the style \url{} uses here:
% (options are: same, rm, sf, tt; defaults to tt)
\urlstyle{same}

% For \email{ADDRESS}, links ADDRESS to the url mailto:ADDRESS
% (uncomment to typeset the e\-/mail address in typewriter font;
%  otherwise, will be typeset in the \urlstyle above)
%\DeclareUrlCommand\emaillink{\urlstyle{tt}}
\providecommand*\emaillink[1]{\nolinkurl{#1}}
\providecommand*\email[1]{\href{mailto:#1}{\emaillink{#1}}}

\providecommand\BibTeX{{B\kern-.05em{\sc i\kern-.025em b}\kern-.08em \TeX}}
\providecommand\Matlab{\textsc{Matlab}}

% Custom hyphenation rules for words that LaTeX has trouble with
%\hyphenation{bio-mim-ic-ry bio-in-spi-ra-tion re-us-a-ble pro-vid-er Media-Wiki}

%%%%%%%%%%%%%%%%%%%%%%%% End Helper Commands %%%%%%%%%%%%%%%%%%%%%%%%%%%

%%%%%%%%%%%%%%%%%%%%%%%%% Begin CV Document %%%%%%%%%%%%%%%%%%%%%%%%%%%%

\begin{document}
\makeheading{Dr.~Casey J.~Law}

\section{Contact Information}

% NOTE: Mind where the & separators and \\ breaks are in the following
%       table. Table is one row made up of three parboxes. The left
%       parbox has address info, the middle parbox has a vertical bar,
%       and the right parbox has phone and electronic contact
%       information.
%
% MACROS: \rcollength is the width of the right column of the table
%             (adjust it to your liking; default is 1.85in).
%         \spacewidth is width of area between left and right boxes.
%
\newlength{\rcollength}\setlength{\rcollength}{1.85in}%
\newlength{\spacewidth}\setlength{\spacewidth}{20pt}
%
\begin{tabular}[t]{@{}p{\textwidth-\rcollength-\spacewidth}@{}p{\spacewidth}@{}p{\rcollength}}%

% Address box
\parbox{\textwidth-\rcollength-\spacewidth}{%
Assistant Project Astronomer\\
Dept of Astronomy and Radio Astronomy Lab\\
University of California, Berkeley\\
Hearst Field Annex\\
Berkeley, CA 94720}

&
% Uncomment to add a vertical bar in middle of contact information
%{\vrule width 0.5pt}
\parbox[m][5\baselineskip]{\spacewidth}{} &

% Non-snail-mail contact information
\parbox{\rcollength}{%
+1-510-859-3636 \\
claw@astro.berkeley.edu}

\end{tabular}

%%
%% In modern CV's, it seems like ``Objective'' is frowned upon. Instead,
%% incorporate it into a well-constructed cover letter. The ``More
%% information'' can go at the end of the CV, but it should not distract
%% from the section giving references available to contact.
%%
%
% \section{Objective}
%
% Placement in an academic position (i.e., faculty, postdoctoral, or
% research scientist) that allows for advanced research in distributed
% complex adaptive systems (i.e., modeling, analysis, design, and
% verification) with a particular focus on the control of engineered
% agents (e.g., for communications, control, software, electronics, and
% sustainability) and the analysis of biological phenomena (e.g.,
% self-organization, ecological rationality)
% \begin{innerlist}
% \item More information and auxiliary documents can be found at\\\url{http://www.tedpavlic.com/facjobsearch/}
% \end{innerlist}

\section{Research Interests}

% highlight radio, x-ray, optical, mm-wave, ir observing experience
% work commissioning novel radio interferometry concepts (low-freq, polarimetry, fast imaging)
% jvla residency built new observing mode with 1 TB/hour data flow
% outreach and teaching in variety of contexts
\textit{Radio interferometry}, fast transients, multiwavelength astronomy, surveys, neutron stars, intergalactic and interstellar media, big data, data science, parallel programming, Python, polarimetry, X-ray astronomy, star clusters

\section{Education}

\textbf{Northwestern University}, Evanston, IL
\begin{outerlist}
\item[] Ph.D., Astrophysics, 2007
        \begin{innerlist}
        \item Thesis Title: \emph{Surveys of the Galactic Center and the Nature of the Galactic Center Lobe}
        \item Research Topics: Galactic outflows, radio, X-ray, and infrared observations of the Galactic center
        \item Adviser: Farhad Yusef-Zadeh
        \end{innerlist}
\end{outerlist}

\halfblankline

\textbf{Boston University}, Boston, MA
\begin{outerlist}
\item[] M.A., Astronomy, 2000
        \begin{innerlist}
        \item Research Topics: Galactic molecular gas survey, optical photometry of stellar clusters, gravitational lensing
        \item Advisers: James Jackson, Ken Janes, Tereasa Brainerd
        \end{innerlist}
\end{outerlist}

\halfblankline

\textbf{University of Hawai`i, Manoa}, Honolulu, HI
\begin{outerlist}
\item[] B.S. with distinction, Physics, 1998
        \begin{innerlist}
        \item Hawai`i Space Grant Fellowship with James Heasley of the Institute for Astronomy
        \item Research Topic: Optical photometry of globular clusters with the UH 2.2m and CFHT
        \end{innerlist}
\end{outerlist}

\section{Employment}

\textbf{Assistant Project Astronomer} \hfill {December 2011 to present}

\textbf{Postdoctoral Fellow} \hfill {January 2009 to November 2011}
\begin{innerlist}

    \item[] Radio Astronomy Lab,
            UC Berkeley
    \begin{innerlist}
      \item Supervisor: Geoff Bower
      \item Leader of VLA fast radio burst project
      \item Radio transients and polarimetry 
      \item Commissioning of the Allen Telescope Array, Karoo Array Telescope, Jansky Very Large Array
    \end{innerlist}

\end{innerlist}

\halfblankline

\textbf{Postdoctoral Fellow} \hfill {July 2006 to December 2008}
\begin{innerlist}

    \item[] Department of Astronomy,
            University of Amsterdam
    \begin{innerlist}
      \item Supervisor: Ralph Wijers
      \item Low frequency radio transients
      \item Commissioning of the Low Frequency Array (LOFAR)
    \end{innerlist}

\end{innerlist}

\halfblankline

\textbf{Astrophysicist} \hfill {September 2000 to June 2002}
\begin{innerlist}

    \item[] Harvard-Smithsonian Center for Astrophysics
    \begin{innerlist}
      \item Data analyst and support astronomer for the \emph{Chandra} X-ray Observatory
      \item Tested, documented, and developed code for the CIAO software package
    \end{innerlist}

\end{innerlist}

\section{Teaching and Outreach}

\textbf{UC Berkeley}, Berkeley, CA
\begin{outerlist}

\item[] \textit{Tutor and Coordinator}, Prison University Project
    \hfill \textbf{2010 -- present}
    \begin{innerlist}
        \item Led tutoring of introductory math for college-level curriculum in San Quentin State Prison.
    \end{innerlist}

\item[] \textit{Co-organizer}, Science@Cal
    \hfill \textbf{2009 -- present}
    \begin{innerlist}
        \item Helped organize and advertise monthly lecture series.
        \item Demonstrated principles of radio astronomy at annual ``Cal Day'' event.
    \end{innerlist}

\item[] \textit{Student Advising}
    \hfill \textbf{2009 -- present}
    \begin{innerlist}
      \item Doctoral students \textbf{Peter Williams}, \textbf{Chat Hull}, and \textbf{James McBride} \\
        Mr.\ Williams, Mr.\ Hull, and Mr.\ McBride used data from the Allen Telescope Array to study radio transients
        and the polarimetric properties of galaxies. Primary adviser: Carl Heiles and Geoff Bower, 2008 -- 2012.
      \item Undergraduate \textbf{Kyle Blanchard} \\
        Mr.\ Blanchard applied a novel radio transient detection algorithm we developed to low-frequency
        radio data from the PAPER radio interferometer.
        Primary adviser: Geoff Bower, 2012 -- 2013.
     \end{innerlist}
\end{outerlist}

\halfblankline

\textbf{University of Amsterdam}, Amsterdam, The Netherlands
\begin{outerlist}

\item[] \textit{Student Advising}
    \hfill \textbf{2006 -- 2007}
    \begin{innerlist}
       \item \textbf{Thijs Coenen}\\
        Master's student at the University of Amsterdam building 
        a machine learning algorithm for the automatic classification 
        of radio transients detected by LOFAR.
        Primary adviser: Ralph Wijers.
        2005.
     \end{innerlist}
\end{outerlist}

\halfblankline

\textbf{Northwestern University}, Evanston, IL
\begin{outerlist}

\item[] \textit{Teaching Assistant} 
    \hfill \textbf{2004}
    \begin{innerlist}
        \item Taught weekly physics discussion session with roughly 100 students.
    \end{innerlist}

\item[] \textit{Observatory Host} 
    \hfill \textbf{2003 -- 2006}
    \begin{innerlist}
        \item Led open night tours of the historic Dearborn Observatory once per month.
    \end{innerlist}
    
\end{outerlist}

\halfblankline

\textbf{Boston University}, Boston, MA
\begin{outerlist}

\item[] \textit{Teaching Assistant} 
    \hfill \textbf{1999 -- 2000}
    \begin{innerlist}
        \item Taught four astronomy lab sections per semester (including night labs).
    \end{innerlist}
\end{outerlist}

\halfblankline

\textbf{University of Hawaii at Manoa}, Honolulu, HI
\begin{outerlist}
\item[] \textit{Co-organizer}, Hawai`i Physics Olympics
    \hfill \textbf{1996 -- 1998}
    \begin{innerlist}
        \item Helped organize annual, state-wide event for high school students.
        \item Designed events to test understanding of physical concepts.
    \end{innerlist}
\end{outerlist}

\begin{outerlist}
\item[] \textit{Co-organizer}, Physics Tutoring
    \hfill \textbf{1996 -- 1998}
    \begin{innerlist}
        \item Created and participated in volunteer physics tutoring service for undergraduates.
    \end{innerlist}
\end{outerlist}

\section{Grants}

\begin{bibsection}

  \item Senior staff: Anomaly detection with fast imaging radio interferometers. University of California Office of the President grant, awarded 2012.

  \item Co-I: A Coherent Transient Detection System for SKA Pathfinders. University of Western Australia Collaboration grant, awarded 2012.

  \item PI: Meeting of LOFAR and the Transient Radio Sky. NWO and NOVA (NL) collaboration support grants, awarded 2008.

  \item PI: Development of a spatio-spectral analysis technique for X-ray data. \emph{Chandra} archival research grant, awarded 2003.

\end{bibsection}

\section{Professional Honors and Service}

\begin{bibsection}

    \item Reviewer for NRAO Science Review Panel and NASA Postdoctoral Program, 2013 -- present

    \item Member of the SKA Transients Science Working Group, 2013 -- present

    \item Referee for the Astrophysical Journal, Astronomical Journal, PASP, and New Astronomy, 2006 -- present

    \item Jansky Very Large Array Resident Observer, 2012

    \item Chair of LOC and member of SOC for ``LOFAR and the Transient Radio Sky'', 2008

    \item Huang Fellowship at Northwestern University, 2002 -- 2003

    \item Two Presidential Fellowships (Research and Teaching) at Boston University, 1998 -- 2000

    \item Four merit-based tuition waivers from the Department of Physics at the University of Hawai`i, 1995 -- 1998

    \item Hawai`i Space Grant Fellowship, 1997

\end{bibsection}

\section{Select Publications}

\begin{bibsection}

	\item ``LOFAR: The LOw-Frequency ARray'', van Haarlem, C. J. et al., 2013, A\&A, 556, 2

	\item ``The RRAT Trap: Interferometric Localization of Radio Pulses from J0628+0909'', Law, C. J. et al., 2012, ApJ, 760, 124

	\item ``All Transients, All the Time: Real-time Radio Transient Detection with Interferometric Closure Quantities'',  Law, C. J. et al., 2012, ApJ, 749, 143

	\item ``Millisecond Imaging of Radio Transients with the Pocket Correlator'',  Law, C. J. et al., 2011, ApJ, 742, 12

	\item ``Spectropolarimetry with the Allen Telescope Array:  Faraday Rotation toward Bright Polarized Radio Galaxies'',  Law, C. J. et al., 2011, ApJ, 728, 57

	\item ``A Constraint on the Organization of the Galactic Center Magnetic Field Using Rotation Measures'',  Law, C. J. et al., 2011, ApJ, 731, 36

	\item ``Observing pulsars and fast transients with LOFAR'',  Stappers, B. W. 2011, A\&A, 530, 80

	\item ``Wild at Heart: The Particle Astrophysics of the Galactic Centre'',  Crocker, R. M. et al., 2011, MNRAS, 413, 763

	\item ``The Allen Telescope Array Pi GHz Sky Survey I. Survey Description and Static Catalog Results for the Bootes Field'',  Bower, G. C. et al., 2010, ApJ, 725, 1792  

	\item ``The Allen Telescope Array Twenty-centimeter Survey -- A 690 sq-deg, 12 Epoch Radio Data Set. I. Catalog and Long-duration Transient Statistics'',  Croft, S. et al. 2010, ApJ, 719, 45

	\item ``A Multiwavelength View of a Mass Outflow from the Galactic Center'',  Law, C. J. 2010, ApJ, 708, 474

	\item ``Green Bank Telescope Multiwavelength Survey of the Galactic Center Region'',  Law, C. J., et al. 2008, ApJS, 177, 255

	\item ``X-Ray Observations of Stellar Clusters Near the Galactic Center'',  Law, C. \& Yusef-Zadeh, F. 2004, ApJ, 611, 858

	\item ``Detection of X-Ray Emission from the Arches Cluster near the Galactic Center'',  Yusef-Zadeh, F., Law, C., et al. 2002, ApJ, 570, 665

\end{bibsection}

\section{Select Invited Talks}

\begin{bibsection}

	\item ``VLA Searches for Fast Radio Transients at 1 TB/hour''. Talk at ``Hotwiring the Transient Universe III'' in Santa Fe, NM, November 2013.

	\item ``Real-Time Transient Detection with Interferometric Closure Quantities''. Talk at ``Interferometric Techniques for Impulsive Signals at Radio Frequencies'' at Ohio State University, April 2013.

	\item ``Real-Time Radio Transient Detection with the VLA''. Seminar for LANL Statistics group, April 2012.

	\item ``The VLA as a Millisecond Transient Survey Machine ''. Colloquium at NRAO Socorro, March 2012.

	\item ``Radio Interferometric Searches for Millisecond Transients''. Colloquium at University of Cape Town, October 2011.

	\item ``Breaking Through the Faraday Fog''. Colloquium at University of Sydney, April 2011.

	\item ``Probing the Transient Radio Sky''. Colloquium at Southampton University, March 2010.

\end{bibsection}

\end{document}
