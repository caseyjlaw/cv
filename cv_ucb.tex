\documentclass[12pt]{article}
\usepackage{url}
\begin{document}

\begin{center}
\Large{Curriculum Vitae -- Casey J. Law}
\end{center}

\begin{flushright}
Radio Astronomy Lab \\
University of California \\
Hearst Field Annex -- B-117 \\
Berkeley, CA, 94720 \\
USA
\end{flushright}

\begin{flushright}
Phone:  +1 510 859 3636 \\
Email:  claw@astro.berkeley.edu \\
Web page: http://astro.berkeley.edu/$\sim$claw
\end{flushright}

\section{Education}
\indent 

Northwestern University, Ph.D., Astrophysics (2007)

Boston University, M.A., Astronomy (2000)

University of Hawai`i, B.S. with distinction, Physics (1998)

\section{Employment}
\indent

January 2009 -- present:  Radio Astronomy Lab Postdoctoral Fellow at UC Berkeley.  Worked with Geoff Bower, Don Backer, and Carl Heiles on radio transients and polarimetry with the Allen Telescope Array and Expanded Very Large Array.

July 2006 -- December 2008: Postdoctoral Fellow at the University of Amsterdam.  Worked with Ralph Wijers and Michael Wise on the commissioning of the Low Frequency Array (LOFAR) for the LOFAR Transient Key Project.

July 2002 -- June 2006:  Graduate Research Assistant.  Performed research with Prof. Farhad Yusef-Zadeh at Northwestern University.  Used X-ray, IR, and radio observations to study many aspects of the Galactic center environment.

January -- June 2004:  Teaching Assistant at Northwestern University.  Taught a weekly physics discussion session with roughly 100 students.

September 2000 -- June 2002:  Astrophysicist at the Harvard-Smithsonian Center for Astrophysics.  Tested and documented the CIAO software package and supported observers for the \emph{Chandra X-Ray Observatory}.

September 1998 -- August 2000:  Performed research as a Presidential University Graduate Fellow at Boston University.  Worked with Prof. James Jackson on the BU Galactic Ring survey, Prof. Kenneth Janes on new analysis of photometry of stellar clusters, and Prof. Teresa Brainerd studying gravitational lensing toward galaxy clusters.

September 1999 -- June 2000:  Taught four astronomy lab sections per semester (incl. night labs) under the Presidential University Teaching Fellowship.

June -- August 1999:  Interned at Textron Systems Kaua`i, developing plans for a new electro-optics site at the Pacific Missile Range Facility on Kaua`i.

July -- August 1998:  Worked with Saroj Sahu in the High Energy Physics Group at the University of Hawai`i on the BELLE project.  Worked on designing and testing data acquisition hardware and software.

September 1997 -- June 1998:  Performed research as a Hawai`i Space Grant Fellow with Prof. James Heasley of the Institute for Astronomy in Honolulu.  Analyzed observations of the globular cluster M71 with the UH 2.2 m and the Canada-France-Hawai`i Telescopes.

\section{Observing Experience}
\indent

Very Large Array: Conducted residency in the study of fast transients with correlated data products. Developing new commensal, real-time transient detection system with the VLA. Attended NRAO Synthesis Imaging Summer School in 2003.

Allen Telescope Array: Led observing programs for polarimetry and transients surveys.  Wrote technical memos and software related to commissioning of array (memos at \url{http://ral.berkeley.edu/ata/memos}).  Organized commissioning effort for two new instruments for fast imaging and pulsar timing.

LOFAR: Commissioned prototype array, Core Station 1, for imaging and calibration.  Proposed for commissioning observations related to transients science.

Green Bank Telescope:  Proposed and conducted polarized continuum observations of the a portion of the Galactic center.  Assisted in observing 3.6, 6, 20, 90 cm continuum surveys of the entire GC region.  Proposed, conducted, and analyzed 6 cm radio recombination line observations.  Attended NRAO Single-Dish Summer School in 2004.

Chandra X-Ray Observatory:  Planned observing proposals with the ACIS instrument and reduced data.  Supported observers while working at the Chandra X-Ray Center.  Awarded archival research grant to develop a spatio-spectral analysis technique for \emph{Chandra} observations.

Spitzer Space Telescope:  Participated in proposal and analysis of survey of the Galactic center with IRAC.

FCRAO 14 m:  Conducted remote observations for a survey of rotational transitions of $^{13}$CO and CS.

Boston University-Lowell Perkins 72 inch:  Observed independently and cooperatively on optical and near-IR projects.

Canada France Hawai`i Telescope:  Analyzed optical data for photometric study of globular clusters.

\section{Professional Honors and Memberships}
\indent

2008:  Chair of Local Organizing Committee and member of Science Organizing Committee for ``LOFAR and the Transient Radio Sky''.  Meeting page at \url{http://www.astro.uva.nl/lofar_transients/workshop/index.html}.

2007:  Best poster prize for session at IAU Symposium 250, ``Massive Stars as Cosmic Engines''.  Poster title:  ``Evidence for a Mini Starburst in the Galactic Center''.

2006 -- Present:   Refereed for the Astrophysical Journal and New Astronomy.

2004 -- 2005:  Awarded two NRAO-GBT Student support grants toward the analysis of GBT polarized continuum surveys and multiwavelength GBT surveys of the Galactic center.

2003:  Awarded archival research grant to develop a spatio-spectral analysis technique for \emph{Chandra} observations.

2002 -- 2003:  Awarded Huang Fellowship at Northwestern University

1998 -- 2000:  Awarded two Boston University Presidential Fellowships for Teaching and Research

1998 -- Present :   Member of American Astronomical Society.

1995 -- 1998:  Awarded four merit-based tuition waivers from the Department of Physics at the University of Hawai`i

1997:  Awarded Hawai`i Space Grant Fellowship

1997:  Member of $\Sigma\Pi\Sigma$, the honor society of the Society of Physics Students

\section{Public Outreach and Education}
\indent

2010 -- present:   Math tutor for the Prison University Project at San Quentin State Prison (\url{http://www.prisonuniversityproject.org}).

2009 -- present:  Helped organize UC Berkeley Astronomy Department outreach.  This included a talk series for the International Year of Astronomy, a broader lecture series as a part of ``Science@Cal'', and the annual ``Cal Day'' outreach event.

2007 -- 2008:  Participated in University of Amsterdam ``Open Dag'', an annual science outreach event.

2003 -- 2006:  Hosted open nights at Northwestern University's historic Dearborn Observatory approximately once per month.

2004:  Teaching Assistant at Northwestern University.  Taught a weekly physics discussion session with roughly 100 students.

1999 -- 2000:  Taught four astronomy lab sections per semester (incl. night labs) under the Presidential University Teaching Fellowship.

1996 -- 1998:  Created and participated in volunteer physics tutoring service at University of Hawai`i.

1996 -- 1998:  Helped organize and design events for the Physics Olympics, a state-wide high school physics competition in Hawai`i.

\section{Refereed Publications}

\begin{enumerate}
\item ``LOFAR detections of low-frequency radio recombination lines towards Cassiopeia A'' Asgekar, A. et al. 2013, A\&A, 551, 11

\item ``VAST: An ASKAP Survey for Variables and Slow Transients'' Murphy, T. et al. 2013, PASA, 30, 6

\item ``ASGARD: A Large Survey for Slow Galactic Radio Transients. I. Overview and First Results'' Williams, P. K. G. et al. 2013, ApJ, 762, 85

\item ``The RRAT Trap: Interferometric Localization of Radio Pulses from J0628+0909'' Law, C. J. et al. 2012, ApJ, 760, 124

\item ``Wide-band simultaneous observations of pulsars: disentangling dispersion measure and profile variations'' Hassall, T. E. 2012, A\&A, 543, 66

\item ``Rapid Development of Interferometric Software Using MIRIAD and Python'', Williams, P. K. G. et al 2012, PASP, 24, 624 

\item ``All Transients, All the Time: Real-time Radio Transient Detection with Interferometric Closure Quantities'' Law, C. J. \& Bower, G. C. 2012, ApJ, 749, 7

\item ``Millisecond Imaging of Radio Transients with the Pocket Correlator'' Law, C. J. et al. 2011, ApJ, 142, 12

\item ``Spectropolarimetry with the Allen Telescope Array:  Faraday Rotation toward Bright Polarized Radio Galaxies'' Law, C. J. et al. 2011, ApJ, 728, 57

\item ``The Allen Telescope Array Pi GHz Sky Survey II. Daily and Monthly Monitoring for Transients and Variability in the Bootes Field'' Bower, G. C. et al., 2011, ApJ, 739, 76

\item ``An automated archival Very Large Array transients survey'' Bell, M. E. et al. 2011, MNRAS, 415, 2

\item ``A Constraint on the Organization of the Galactic Center Magnetic Field Using Rotation Measures'' Law, C. J. et al., 2011, ApJ, 731, 36

\item ``Observing pulsars and fast transients with LOFAR'' Stappers, B. W. 2011, A\&A, 530, 80

\item ``Wild at Heart:-The Particle Astrophysics of the Galactic Centre'' Crocker, R. M. et al., 2011, MNRAS, 413, 763

\item ``The Allen Telescope Array Twenty-centimeter Survey—A 700-square-degree, Multi-epoch Radio Data Set. II. Individual Epoch Transient Statistics'' Croft, S. et al. 2011, ApJ, 731, 34

\item ``Gamma-Rays and the Far-Infrared-Radio Continuum Correlation Reveal a Powerful Galactic Center Wind'' Crocker, R. M. et al., 2010, MNRAS, 411L, 11

\item ``The Allen Telescope Array Pi GHz Sky Survey I. Survey Description and Static Catalog Results for the Bootes Field'' Bower, G. C. et al., 2010, ApJ, 725, 1792  

\item ``Primary Beam Shape Calibration from Mosaicked, Interferometric Observations'' Hull, C. L. H. et al., 2010, PASP, 122, 1510

\item ``The Allen Telescope Array Twenty-centimeter Survey—A 690 deg2, 12 Epoch Radio Data Set. I. Catalog and Long-duration Transient Statistics'' Croft, S. et al. 2010, ApJ, 719, 45

\item ``A Multiwavelength View of a Mass Outflow from the Galactic Center'' Law, C. J. 2010, ApJ, 708, 474

\item ``Radio Recombination Lines Toward the Galactic Center Lobe'' Law, C. J., Backer, D., Yusef-Zadeh, F., \& Maddalena, R. 2009, ApJ, 695, 1070

\item ``A Catalog of X-Ray Point Sources from Two Megaseconds of Chandra Observations of the Galactic Center'' Muno, M. P. et al. 2009, ApJS, 181, 110

\item ``X-Ray Observations of the Sagittarius D HII Region toward the Galactic Center with Suzaku'' Sawada, M. et al. 2009, PASJ, 61S, 209

\item ``Comparison of 3.6-8.0 μm Spitzer/IRAC Galactic Center Survey Point Sources with Chandra X-Ray Point Sources in the Central 40 by 40 Parsecs'' Arendt, R. G. et al. 2008, ApJ, 685, 958

\item ``A Wide-Area VLA Continuum Survey near the Galactic Center at 6 and 20 cm Wavelengths'' Law, C. J., Yusef-Zadeh, F., \& Cotton, W. D. 2008, ApJS, 177, 515

\item ``Green Bank Telescope Multiwavelength Survey of the Galactic Center Region'' Law, C. J., Yusef-Zadeh, F., Cotton, W. D., \& Maddalena, R. J. 2008, ApJS, 177, 255

\item ``The Mid-Infrared Colors of the Interstellar Medium and Extended Sources at the Galactic Center'' Arendt, R. G. et al. 2008, ApJ, 682, 384

\item ``The Cool Supergiant Population of the Massive Young Star Cluster RSGC1'' Davies, B. et al. 2008, ApJ, 676, 1016

\item ``Diffuse, Nonthermal X-Ray Emission from the Galactic Star Cluster Westerlund 1'' Muno, M. P., Law, C., et al. 2006, ApJ, 650, 203

\item ``A Neutron Star with a Massive Progenitor in Westerlund 1'' Muno, M. P. et al. 2006, ApJ, 636, L41

\item ``The Nature of Nonthermal X-Ray Filaments Near the Galactic Center'' Yusef-Zadeh, F., Wardle, M., Muno, M., Law, C., \& Pound, M. 2005, AdSpR, 35, 1074

\item ``X-Ray Observations of Stellar Clusters Near the Galactic Center'' Law, C. \& Yusef-Zadeh, F. 2004, ApJ, 611, 858

\item ``Nonthermal Emission from the Arches Cluster (G0.121+0.017) and the Origin of Gamma-Ray Emission from 3EG J1746-2851'' Yusef-Zadeh, F., Nord, M., Wardle, M., Law, C., Lang, C., \& Lazio, T. J. W. 2003, ApJ, 590, 103

\item ``Detection of X-Ray Emission from the Arches Cluster near the Galactic Center'' Yusef-Zadeh, F., Law, C., Wardle, M., Wang, Q. D., Fruscione, A., Lang, C. C., \& Cotera, A. 2002, ApJ, 570, 665

\item ``The Origin of X-Ray Emission from a Galactic Center Molecular Cloud:  Low-Energy Cosmic-Ray Electrons'' Yusef-Zadeh, F., Law, C., \& Wardle, M. 2002, ApJ, 568, L121

\item ``A Comparison of $^{13}$CO and CS Emission in the Inner Galaxy'' McQuinn, K.B.W., Simon, R., Law, C. J., Jackson, J. M., Bania, T. M., Clemens, D. P., \& Heyer, M. H. 2002, ApJ, 576, 274
\end{enumerate}

\section{Seminars and Professional Talks}
\indent

\begin{enumerate}

\item ``Interferometric Studies of Millisecond Transients''  Seminar at NRAO--Socorro, July 2011.

\item ``Breaking Through the Faraday Fog''  Seminar at University of Sydney, April 2011.

\item ``Breaking Through the Faraday Fog''  Seminar at Radio Astronomy Lab, UC Berkeley, May 2010.

\item ``Probing the Transient Radio Sky''  Seminar at Max Planck Institut f\"ur Kernphysik, April 2010.

\item ``Probing the Transient Radio Sky''  Seminar at Southampton University, March 2010.

\item ``Polarimetry with the Allen Telescope Array''  Talk at SKA Science and Engineering meeting at University of Manchester, March 2010.

\item ``Observing the Transient Radio Sky with the Allen Telescope Array''  Talk at ``Hotwiring the Transient Universe II'' in Santa Cruz, CA, April 2009.

\item ``Observing Radio Transients''  Theory Lunch talk in UC Berkeley, April 2009.

\item ``Early Constraints on Transients with LOFAR''  Talk at ``LOFAR and the Transient Radio Sky'', a December 2008 workshop in Amsterdam, Netherlands.

\item ``Early Constraints on Transients with LOFAR''  Talk at ``Astrophysics with E-LOFAR'', a September 2008 workshop in Hamburg, Germany.

\item ``Probing the Transient Radio Sky''  Seminar at Radio Astronomy Lab, UC Berkeley, July 2008.

\item ``Early Science with LOFAR Transients''  Talk at ``Low Frequency Pulsar Science'', a June 2008 workshop in Leiden, Netherlands.

\item ``A Survey for Transient Sources with LOFAR/CS1''  Talk at ``Wide-field monitoring of the dynamic radio sky'', a June 2007 workshop in Kerastari, Greece.

\item ``A Study of the Galactic Center Lobe''  Dissertation talk at 207th AAS meeting (103.05), Washington D.C., January 2006.

\item ``Evidence for a Mass Outflow from the Galactic Center?''  Talk presented at the NRAO Array Operations Center in Socorro, NM, August 2005.

\item ``Internal Irradiation of the Sgr B2 Molecular Cloud''  Talk presented at the ``High Energy Phenomena in the Galactic Center'' meeting in Paris, France, June 2005.

\item ``X-Ray Emission from Stellar Clusters Near the Galactic Center''  Talk presented at ``Galactic Center Workshop 2002'' in Hawai`i, 2002.

\item ``Analyzing the Relative Ages of Thick Disk Globular Clusters''  Talk presented at the Hawai`i Space Grant consortium meeting in Honolulu, HI, 1998.
\end{enumerate}

\end{document}
